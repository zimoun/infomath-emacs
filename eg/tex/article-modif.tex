\documentclass[11pt]{article}
\usepackage[utf8]{inputenc}
\usepackage[T1]{fontenc}
\usepackage{amsmath,amssymb,amsfonts}

\author{John Doe}
\date{\today}
\title{Materials from IEEE Author package}

\begin{document}
\maketitle

\begin{abstract}
These instructions give you guidelines for preparing manuscripts for
IEEE Transactions and Journals. Use this document as a template if you are
using \LaTeX. Otherwise, use this document as an
instruction set. The file of your manuscript will be formatted further
at IEEE. Manuscript titles should be written in uppercase and lowercase letters,
not all uppercase.
\end{abstract}


\section{Introduction}
\label{sec:introduction}

If your manuscript is intended for a conference, please contact your
conference editor concerning acceptable word processor formats for your
particular conference. IEEE will do the final formatting of your
manuscript. If your manuscript is intended for a Conference, please observe the
conference page limits.


\subsection{Abbreviations and Acronyms}
\label{sec:abrev}

Define abbreviations and acronyms the first time they are used in the text,
even after they have already been defined in the abstract. Abbreviations
incorporate periods should not have spaces: write ``C.N.R.S.,'' not ``C. N.
R. S.'' Do not use abbreviations in the title unless they are unavoidable
(for example, ``IEEE'' in the title of this article).

The Stoke's theorem reads,
\begin{equation}
  \label{eq:beautiful}
  \int_{\partial\Omega} \omega = \int_\Omega d\omega
\end{equation}
which means,
\begin{equation}
  \label{sec:ugly}
  \int_\Gamma \mathbf{F}\cdot d\mathbf{\Gamma}
  =
  \int\int_S \nabla\times\mathbf{F}\cdot dS
\end{equation}
and everybody uses it because it rocks!


\subsection{Other Recommendations}
\label{sec:reco}

Use one SPACE after periods and colons. Hyphenate complex modifiers:
``zero-field-cooled magnetization.'' Avoid dangling participles, such as,
``Using \eqref{eq:einstein}, the potential was calculated.'' [It is not clear who or what
used \eqref{eq:einstein}.] Write instead, ``The potential was calculated by
using \eqref{eq:einstein},'' or
``Using \eqref{eq:einstein}, we calculated the potential.''


\section{Where I type stuff}
\label{sec:my-stuff}

Number equations consecutively with equation numbers in parentheses
flush with the right margin, as in \eqref{eq:einstein}. Punctuate
equations when they are part of a sentence, as in
\begin{equation}\label{eq:einstein}
  E=mc^2.
\end{equation}
Then do not forget to cite \cite{Saad2002_book}.

\subsection{The table part}
\label{sec:fig}

Be aware of the different meanings of the homophones ``affect'' (usually a
verb) and ``effect'' (usually a noun), ``complement'' and ``compliment,''
``discreet'' and ``discrete,'' ``principal'' (e.g., ``principal
investigator'') and ``principle'' (e.g., ``principle of measurement''). Do
not confuse ``imply'' and ``infer.'' See \cite{getdp-ieee1999}.
\begin{table}[!h]
  \centering
  \caption{My table}
  \label{tab:student}
  \begin{tabular}{ll|c}
    Name & City & Editor \\
    \hline
    Berty & Vesouls & VSCode \\
    Simon & Montbazens & Emacs \\
    You & Paris & Emacs?
  \end{tabular}
\end{table}
Prefixes such as ``non,'' ``sub,'' ``micro,'' ``multi,'' and ``ultra'' are
not independent words; they should be joined to the words they modify,
usually without a hyphen. There is no period after the ``et'' in the Latin
abbreviation ``\emph{et al.}'' (it is also italicized). The abbreviation ``i.e.,'' means
``that is,'' and the abbreviation ``e.g.,'' means ``for example'' (these
abbreviations are not italicized).


\subsection{Other stuff}
\label{sec:other}

Please use ``soft'' (e.g., \verb|\eqref{eq:einstein}|) cross references instead
of ``hard'' references (e.g., \verb|(1)|). That will make it possible
to combine sections, add equations, or change the order of figures or
citations without having to go through the file line by line.

\bibliographystyle{plain}
\bibliography{the}

\end{document}
